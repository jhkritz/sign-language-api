\documentclass{beamer}
\mode<presentation>
\title{Project Demonstration: Sign Language API}
\author{TODO: add authors}
\date{}
\usepackage{natbib}

\begin{document}

\begin{frame}[t]
	\titlepage
\end{frame}

\begin{frame}[t]
	\frametitle{Introduction}
	\begin{itemize}
		\item In this presentation we will discuss our project.
		\item We will also have more than one bullet point on this slide, probably.
	\end{itemize}
\end{frame}

\begin{frame}[t]
	\frametitle{Classification accuracy}
	\framesubtitle{Initial results}
	\begin{figure}
		\includegraphics[width=\textwidth]{initial_classification_performance.png}
	\end{figure}
	\begin{itemize}
		\item These results occurred when $K$ was set to an improper value.
	\end{itemize}
\end{frame}

\begin{frame}[t]
	\frametitle{Classification accuracy}
	\framesubtitle{After adjusting k}
	\begin{figure}
		\includegraphics[width=\textwidth]{k_adjusted.png}
	\end{figure}
\end{frame}

\begin{frame}[t]
	\frametitle{Classification accuracy}
	\framesubtitle{KNN after PCA}
	\begin{figure}
		\includegraphics[width=\textwidth]{knn_pca.png}
	\end{figure}
	\begin{itemize}
		\item Images are projected onto at most 10 features.
		\item Without PCA, each image would consist of $120000$ features.
	\end{itemize}
\end{frame}

\begin{frame}[t]
	\frametitle{Classification accuracy}
	\framesubtitle{Using Linear Discriminant Analysis}
	\begin{figure}
		\includegraphics[width=\textwidth]{lda.png}
	\end{figure}
	\begin{itemize}
		\item LDA maximises between class separation.
		\item It also assumes all classes have the same variance.
		\item It also projects onto at most $N-1$ features,
			where $N$ is the number of classes.
	\end{itemize}
\end{frame}

\begin{frame}[t]
	\frametitle{Process}
	\framesubtitle{Issue boards}
	\begin{figure}
		\includegraphics[width=\textwidth]{issue_boards.png}
		\caption{A screenshot of our issue boards for sprint two.}
	\end{figure}
\end{frame}

%\begin{frame}[t]
%	\frametitle{This slide has a figure in it}
%	\begin{columns}
%	\begin{column}{0.5\textwidth}
%		\begin{figure}
%			\includegraphics[width=\textwidth]{missing_figure.png}
%			\caption{This is the caption of the figure}
%			\label{fig:missing_figure}
%		\end{figure}
%		\end{column}
%		\begin{column}{0.5\textwidth}
%			\begin{itemize}
%				\item According to \cite{me} there is a figure to the left of this bullet point.
%				\item Additionally, this is another bullet point.
%			\end{itemize}
%		\end{column}
%	\end{columns}
%\end{frame}

\begin{frame}[t]
	\frametitle{References}
	\bibliographystyle{plainnat}
	\bibliography{bibl}
\end{frame}

\end{document}
